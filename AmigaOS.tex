\documentclass{beamer}

\mode<presentation>
{
  \usetheme{Dresden}
  \setbeamercovered{transparent}
}

\usepackage[english]{babel}

\usepackage{listings}
\usepackage{hyperref}

\title [Amiga OS] % short title
{Amiga Operating System}

\author[Sweta] % (short authors list)
{Sweta Pandya}

\institute[Loyola University Chicago] % (optional, but mostly needed)
{
  % \inst{1}%
    Department of Computer Science \\
  Loyola University Chicago \\
}


\begin{document}

\begin{frame}
\begin{center}
    \includegraphics[height=3cm]{diagrams/Amiga_Logo}
  \end{center}
  \titlepage
\end{frame}

% SLIDE

\begin{frame}{Outline}
  \begin {itemize}
	\item History
	\item Components
	\item Technical Overview
	\item Semaphore
	\item Scheduling
	\item Deadlock
	\item Future
	\item References
 \end{itemize}
\end{frame}


% SLIDE
\begin{frame}{History}
  \begin{itemize}
    \item Amiga OS was developed first by Commodore International, and initially introduced in 1985 with the Amiga 		1000.
     \item Amiga DOS and Workbench.
     \item In 2001 they contracted Amiga OS 4 development to Hyperion Entertainment and in 2009 they granted
	Hyperion an exclusive, worldwide right to Amiga OS 3.1 in order to develop and market Amiga OS 4 and 
	subsequent versions.
    \end{itemize}
 \end{frame}


% SLIDE
\begin{frame}{Components}
  \begin{itemize}
    \item Kickstart
      \item Amiga DOS
      \item Workbench
    \end{itemize}
\end{frame}


% SLIDE
\begin{frame}{Kickstart}
  \begin{itemize}
  	\item Bootstrap firmware.
      	\item Its purpose is to initialize the Amiga hardware and core components of Amiga OS and then attempt to boot from a bootable volume, such as a      
       floppy disk or hard disk drive.
      \item The first production Amiga, the Amiga 1000, required Kickstart to be loaded first from floppy disk into RAM reserved for this purpose.
       Later models generally hold Kickstart on an embedded ROM chip, improving start-up times.
    \end{itemize}
\end{frame}
 

% SLIDE
\begin{frame}{Amiga DOS}
  \begin{itemize}
      	\item In Amiga OS 1.x, Amiga DOS was written in BCPL. From Amiga OS 2.x onwards, Amiga DOS was rewritten in C.

      \item The console had various features,  such as command history, pipelines, and automatic creation of files when output was redirected.
\item Feature of "script" programming.
\item The first file system was simply called Amiga FileSystem. It was soon replaced by FastFileSystem (FFS).
\end{itemize}
\end{frame}



% SLIDE
\begin{frame}{Workbench}
  \begin{itemize}
  	\item Graphical Interface for Amiga DOS.
\begin{itemize}
      	\item Tools
	\item Tool types
	\item Drawers
	\item Projects
\end{itemize}
\item Workbench 1.x
\item Workbench 2.x
\item Workbench 3.x
\item Workbench 4.x
\end{itemize}
\end{frame}


% SLIDE
\begin{frame}{Workbench 1.x}
 
  \begin{center}
    \includegraphics[height=4cm]{diagrams/Amiga_Workbench_1_0}
  \end{center}
 
\begin{itemize}
  	\item Amiga1000
	\item Maximum resolution 640 x 512 
	
\end{itemize}
\end{frame}

% SLIDE
\begin{frame}{Workbench 2.x}
 
  \begin{center}
    \includegraphics[height=4cm]{diagrams/Amiga_Workbench_2_0}
  \end{center}
 
\begin{itemize}
  	\item No unified look and feel design 
	\item Amiga Guide
	
\end{itemize}
\end{frame}


% SLIDE
\begin{frame}{Workbench 3.x}
 
  \begin{center}
    \includegraphics[height=3cm]{diagrams/Amiga_Workbench_3_9}
  \end{center}
 
\begin{itemize}

  	\item Changed the complete look and feel of its interface 
	\item Support for New Icons and various other third party GUI enhancements to improve elder Amiga interfaces 
\item Introduction of the AmiDOCK, a program start bar
\item Amiga 1200 and Amiga 4000 

\end{itemize}
\end{frame}

% SLIDE
\begin{frame}{Workbench 4.x}
 
  \begin{center}
    \includegraphics[height=4cm]{diagrams/Amiga_Workbench_4_0}
  \end{center}
 
\begin{itemize}
  	\item Fully PowerPC compatible
	\item Drag and drop of Workbench icons
	\item TrueType fonts and movie player named "Action" with DivX and MPEG-4 support
\end{itemize}
\end{frame}

% SLIDE
\begin{frame}{Technical Overview}
  \begin{itemize}
  	\item Work with Motorola Processors
	\begin{itemize}
	\item A1000 - 68EC020
	\item A4000/030 - 68EC030
\end{itemize}

\item Process States 

\begin{itemize}
  	\item Two queues 
	
\end{itemize}
\item Memory Management

\begin{itemize}
  	\item No memory protection
	\item Problem of memory corruption
	
\end{itemize}
\item Exec Microkernel

\begin{itemize}
  	\item Uses much less memory than other operating systems that offer the same features.

	\item Deals with the resource handling
\end{itemize}

\end{itemize}
\end{frame}


% SLIDE
\begin{frame}{Semaphores}
  \begin{itemize}
  	\item Forbid() and Disable() function to turn off interrupts
	\item Exec semaphore functions:
\begin{itemize}
\item AddSemaphore()
\item AttemptSemaphore()
\item FindSemaphore()
\item InitSemaphore()
\item ObtainSemaphore()
\item ObtainSemaphoreList(),
\item ObtainSemaphoreShared()
\item ReleaseSemaphore()
\item ReleaseSemaphoreList()
\item RemSemaphore()
\end{itemize}
\end{itemize}
\end{frame}




% SLIDE
\begin{frame}{Scheduling}
  \begin{itemize}
  	\item Pre-emptive Multitasking
\item Special feature of allowing the user to have some control over process scheduling
\begin{itemize}
\item Control Panel
\item ChangeTaskPriority Command
\item Executive Tool

\end{itemize}
\item Advantages
\begin{itemize}
\item Runs number of programs in the background at a lower priority than other applications
\item Object oriented programming can be implemented
\end{itemize}
\item Disadvantages
\begin{itemize}
\item Single program can hog the processor and lock out other programs
\item Ability for one program to
overwrite or corrupt another
\end{itemize}
\end{itemize}
\end{frame}



% SLIDE
\begin{frame}{Deadlock}
  \begin{itemize}
  	\item Ignore the deadlock problem
\item Few provisions made for deadlock avoidance and the find-and-repair technique
\end{itemize}
\end{frame}

% SLIDE
\begin{frame}{Future??}
  \begin{itemize}
  	\item After the release of Amiga’s newest OS, 4.0, Amiga will continue to develop and
improve upon the initial version. The future of Amiga’s commitment hinges on the
success of these developing OS’s. If they are a success then we will mostly likely see
development of an OS 4.5 and OS 5.

\end{itemize}
\end{frame}



% SLIDE
\begin{frame}{References}
  \begin{itemize}
  	\item https://users.cs.jmu.edu/abzugcx/public/Student-Produced-Term-Projects/Operating-Systems-2003-FALL/Amiga-by-Kevin-Marinak-Jon-Bradley-Holly-Medeiros-John-Feehan-Nimish-Patel.pdf
\item http://en.wikipedia.org/wiki/AmigaOS
\item http://www.amigahistory.co.uk/caos.html
\item http://www.amigaforever.com/
\end{itemize}
\end{frame}


% SLIDE
\begin{frame}{}
 
  \begin{center}
    \includegraphics[height=4cm]{diagrams/Thank_You.jpg}
  \end{center}
\end{frame}

\end{document}